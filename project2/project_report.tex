\documentclass[fontsize=11pt]{article}
\usepackage{amsmath}
\usepackage[utf8]{inputenc}
\usepackage[margin=0.75in]{geometry}

\title{CSC111 Project 2 Final Report: Enhancing Business Success Through Geospatial and Review Data Analysis}
\author{Ali Towaiji, Tanay Langhe}
\date{\today}

\begin{document}
\maketitle

\section*{Introduction}

Our project dives into the intricate world of business success, focusing on unraveling the factors that significantly impact a business's prosperity in various geographic locations. Rooted in the initial curiosity about what sets successful businesses apart, we refined our research question to: By setting criteria and showing which business meet such criteria is it possible to determine the factors that the business have to achieve said criteria? This question and our program is the first step of many to answering this question since we allow the user to clearly filter and limit what they are looking for in a business, which allows for further research to be done on the business that do pass the users criteria.

Feedback Incorporation and Scope Adjustment:

In response to the feedback on our proposal, we worked to enhance our creativity and complexity by making our project more interactive. We did so by expanding our computational plan to be able to receive requirements from users in order to present specific data. We also utilized Plotly as well as pandas in order to illustrate the desired businesses pinpointed to their exact location on a mapbox (that required an API). While also keeping the central role of graphs in this project as we created a scoring system depending on how similar a business is to the users' requirements, where the higher the score, the more likely this is a business the user is interested in (graph is 3d, kind of, maybe we can use that).

\section*{How to use}

We will attach for you the test files of the data that may be used and the test files for faster processing. They are called "meta-test.json" and "meta-test2.json". First, when you run the program, you will be given questions of which states you want to check and an example of how they should look when you type them. We will send you a zip file with all states data but recommend you to use ``test, test2'' (both or each one alone but test is more rich in data) in order to use the test files (do not add quotations). After that, it will ask you for a minimum rating and review number, which you may enter any realistic number as the data is real as well. After that, you may add a certain category that you want to check (this can be Convenience store or Gas station please use these as they give you the best illustration, but any category will work as long as you grab the category name from the database) then you will be given a number of the amount of businesses that apply to this criteria then a visualization of these businesses and scores embedded in their nodes and color coding.

\section*{Datasets}

We utilized the Google Local Reviews (2021) dataset, which encompasses businesses all over the United States of America. The dataset was composed of detailed business attributes like names, addresses, categories, ratings, google maps id, price, hours, state and reviews, which all play a major role in showing a business's health, relevance, and popularity. To ensure that our data was clean and accurate we applied rigorous cleaning and filtering in order to remove any business with unreliable and non-comprehensive data. This removed things like google maps id which was useless, and price which were all null. This data set showed its potential in offering rich insights on how businesses are intertwined with the location they reside in. It also showed different customer perspectives in different geographies. The data files that we utilized were JSON files for each American state and through filtering, the user may choose only some states as needed. We also have 2 test files that help visualize our project in a more controlled environment in order to showcase our program's potential.

\section*{Computational Overview}

Data Representation and Analysis: We first worked through the datasets and cleaned and filtered them to match the criteria needed to be reliable using the parse function:

\begin{itemize}
    \item \textbf{parse}: runs through the datasets and cleans them, removing unnecessary keys in each dictionary and only returning what is useful for research in a simplified format.
    \item \textbf{get\_states}: asks the user for input on which states they would like to visualize.
    \item \textbf{get\_criteria}: asks the user for certain criteria such as minimum average rating, minimum number of reviews.
\end{itemize}

We constructed a graph where businesses were the vertices, and the edges between them represented relations based on location, average rating, number of reviews, and similar categories. We utilized the graph to examine the interrelations of businesses, the influence of geographical location and categorical qualities on the businesses and their success. We used a graph class and a vertex class like in this course and made many helper functions that play in role in creating vertices and adding the edges between them:

\begin{itemize}
    \item \textbf{add\_vertex}: adds vertices based on the location of the business as there can be duplicate businesses in different locations.
    \item \textbf{build\_edges}: builds the edges between the vertices by a certain criteria using a helper function "is\_similar".
    \item \textbf{is\_similar}: checks if the ratings, reviews, and categories match in certain limits or qualities.
    \item \textbf{compute\_scores}: computes the score based on a certain formula that takes into account the number of reviews, average rating, and the number of neighbors.
    \item \textbf{filter\_data}: filters the data based on category, review number, rating with respect to user inputs
\end{itemize}

We visualized the information received from the graph in a real-life map where each business is pinpointed in its exact location in the real world. The visualization also shows a scoring system with proper color coding in order to distinguish which businesses are closest to the chosen criteria. This is done by using the latitude and longitude from the metadata to place the location and the size and score of each node is done by using a calculated score.



\section*{Project Evolution}

Reflecting on our proposal, our project evolved significantly in terms of creativity and complexity. Based on the TA feedback, we explored ways to enhance our computational complexity and found ways to make the program more interactive for the user. We made it as interactive as needed in order for the user to receive the exact data needed based on their criteria. Instead of only utilizing many graphs that showed different similarities between businesses, we worked to change that into what is basically a database where you enter requirements and receive information based on your needs. All this while also displaying them in a creative scene where businesses are shown in their exact position on a real-world map instead of a plain graph.

\section*{Discussion and Future Directions}

Our findings show the relationships between businesses and their success based on ratings and reviews with respect to their geographical position. Our analysis provides valuable insight as it serves as basically a database for finding the best businesses in certain categories using an intricate scoring system that scores based on the relations of all businesses.

In the future, we could work to find more detailed datasets that give much more information regarding and relating to what makes a business successful, and we can work to make it more interactive by adding those extra keys that may be used. We can also explore different analytical models in order to give more visualization techniques and offer more personalized insights into business success dynamics.

\section*{Conclusion}

This project stands as a testament to the power of computational analysis in unraveling the complexities of business success across geographies. Through sophisticated data manipulation, algorithmic analysis, and interactive visualization, we offer stakeholders actionable insights, fostering informed decision-making for enhancing business outcomes.

\section*{References}

Li, Jiacheng, et al. "UCTopic: Unsupervised Contrastive Learning for Phrase Representations and Topic Mining." 2022.

Yan, An, et al. "Personalized Showcases: Generating Multi-Modal Explanations for Recommendations."

\end{document}
