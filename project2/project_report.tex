Final Report: Enhancing Business Success Through Geospatial and Review Data Analysis
Group Members:
Ali Towaiji
Tanay Langhe
Introduction
Our project dives into the intricate world of business success, focusing on unraveling the factors that significantly
impact a business's prosperity in various geographic locations. Rooted in the initial curiosity about what sets
successful businesses apart, we refined our research question to: "What factors contribute most to the success of
businesses in a given area?" This exploration is not only about distinguishing successful businesses from the rest but
also about understanding the broader economic implications such as job creation, innovation, and economic growth.

Feedback Incorporation and Scope Adjustment
Responding to feedback on our proposal, we expanded our computational plan to not only leverage graph data structures
for modeling business relationships but also to introduce a comparative analysis of different analytical algorithms.
This approach aimed at deepening the project's computational complexity and creativity.

Datasets
Our study utilizes the Google Local Reviews (2021) dataset, encompassing rich business attributes like names,
addresses, categories, ratings, and reviews. To ensure relevance and accuracy, we implemented rigorous data cleaning
protocols, focusing on businesses with comprehensive and reliable data. This dataset's choice was motivated by its
potential to offer nuanced insights into the interplay between business performance and customer perceptions across
different geographies.

Computational Overview
Data Representation and Analysis
Adhering to our computational plan, we constructed a graph where businesses served as nodes, and various attributes,
such as proximity and category, formed the edges. This structure facilitated the examination of business interrelations
and the influence of geographical and categorical proximities on success metrics.

Algorithmic Innovations
Beyond basic graph analysis, we embarked on implementing and comparing different algorithms to dissect the dataset's
depth. This involved:

Clustering algorithms to identify business success patterns.
Sorting algorithms for organizing data, enabling a nuanced understanding of factors driving business success.

Visualization Techniques
Leveraging Plotly, we developed interactive, multidimensional graphs showcasing the complex relationships between
businesses based on our analytical findings. This not only provided a visual representation of our computational
analysis but also offered users an engaging way to explore the data.

Program Execution
Our project, encapsulated in Python, requires the installation of specific libraries listed in requirements.txt,
including Plotly for visualization and NetworkX for graph analysis. Following dataset preprocessing, the main.py
script orchestrates the analysis, culminating in interactive visualizations that highlight key success factors.

Project Evolution
Reflecting on our journey from proposal to final submission, our project's scope and depth evolved significantly.
Incorporating TA feedback, we enhanced the computational complexity and explored creative data analysis avenues,
ensuring our exploration remained grounded in rigorous computational methods.

Discussion and Future Directions
Our findings reveal intricate relationships between business success and various factors, with geographical location
and customer reviews emerging as significant. While our analysis provides valuable insights, we acknowledge limitations
such as dataset comprehensiveness and the challenge of capturing all potential success factors.

Future work could expand the dataset, explore additional analytical models, and refine the interactivity of
visualizations to offer more personalized insights into business success dynamics.

Conclusion
This project stands as a testament to the power of computational analysis in unraveling the complexities of business
success across geographies. Through sophisticated data manipulation, algorithmic analysis, and interactive
visualization, we offer stakeholders actionable insights, fostering informed decision-making for enhancing
business outcomes.

References
Li, Jiacheng, et al. "UCTopic: Unsupervised Contrastive Learning for Phrase Representations and Topic Mining." 2022.
Yan, An, et al. "Personalized Showcases: Generating Multi-Modal Explanations for Recommendations."
